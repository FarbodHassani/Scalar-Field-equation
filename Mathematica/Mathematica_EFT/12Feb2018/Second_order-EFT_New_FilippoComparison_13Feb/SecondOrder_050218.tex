\documentclass[a4paper,12pt]{article}
%% My standard included packages
%\pdfoutput=1 % if your are submitting a pdflatex (i.e. if you have
%             % images in pdf, png or jpg format)
%\usepackage{jcappub} % for details on the use of the package, please
%                     % see the JCAP-author-manual
%\usepackage[T1]{fontenc} % if needed

\usepackage{setspace}           % Allows easy changes to line spacing 
\usepackage{graphicx}           % Allows including of graphics files
\usepackage{amsmath}            % Additional math capabilities
\usepackage{marginnote}         % Used with todonotes package
\usepackage{datetime}           % Allows formatting of date and time
\newcommand {\be}{\begin{equation}}
\newcommand {\ee}{\end{equation}}

\usepackage{enumitem} 
\usepackage{listings}
\usepackage{amsmath}
\usepackage{graphicx}% Use pdf, png, jpg, or eps� with pdflatex; use eps in DVI mode
\usepackage{caption}
\usepackage{subcaption}
          % List formatting commands
\setlist{noitemsep}             % Remove space between list items 
%\usepackage{subfigure}          % Create numbered and captioned subfigures
\usepackage{rotating}           % Create landscape tables and figures
\usepackage[dvipsnames]{xcolor} % Refer to colors by name
\usepackage[colorlinks=true,urlcolor=blue,linkcolor=Orange,citecolor=RedViolet]{hyperref}           % URLS and hyperlinks
%\usepackage{hyperref}           % URLS and hyperlinks
\usepackage{float}              % Activate [H] option to place figure HERE
\usepackage[numbers]{natbib}
\usepackage{versionPO}          % Include text conditionally
\usepackage{caption}
%\usepackage[utf8]{inputenc}
%\usepackage[nottoc]{tocbibind}
\lstset{basicstyle=\ttfamily,
  showstringspaces=false,
  commentstyle=\color{red},
  keywordstyle=\color{blue}
}
% These next lines allow including or excluding different versions of text
% using versionPO.sty
\includeversion{notes}		% Include notes?
%\excludeversion{notes}
\excludeversion{comment}
\includeversion{links}          % Turn hyperlinks on?
\excludeversion{submit}		% Format for conference submission?
\includeversion{toc}		% Include table of contents?
%\graphicspath{{./Results1-Perihelionadvance}}

% Turn off hyperlinking if links is excluded
\iflinks{}{\hypersetup{draft=true}}

% Notes options
\ifnotes{%
\usepackage[margin=1in,paperwidth=10in,right=2.5in]{geometry}%
\usepackage[textwidth=1.4in,shadow,colorinlistoftodos]{todonotes}%
}{%
\usepackage[margin=1in]{geometry}%
\usepackage[disable]{todonotes}%
}

% Allow todonotes inside footnotes without blowing up LaTeX
% Next command works but now notes can overlap. Instead, we'll define 
% a special footnote note command that performs this redefinition.
%\renewcommand{\marginpar}{\marginnote}%

% Save original definition of \marginpar
\let\oldmarginpar\marginpar
% Workaround for todonotes problem with natbib (To Do list title comes out wrong)
\makeatletter\let\chapter\@undefined\makeatother % Undefine \chapter for todonotes
% Packages included specifically for this document.
\usepackage{texintro}           % Document-specific definitions
\usepackage{tocvsec2}           % More flexible formatting of table of contents
\usepackage{bibentry}           % Print full citation in text
\nobibliography*                                % Allow use of \bibentry command
\usepackage{tikz}             % Already included by todonotes
\usetikzlibrary{matrix}
\usepackage[retainorgcmds]{IEEEtrantools}  % Equation formatting. Option needed to
                                           % allow enumitem to work.

% Workaround for todonotes problem with natbib (To Do list title comes out wrong)
% If you're including tocvsec2, do so before this command.
\makeatletter\let\chapter\@undefined\makeatother % Undefine \chapter for todonotes.

% Number paragraphs and subparagraphs and include them in TOC
\setcounter{tocdepth}{2}

\usepackage[affil-it]{authblk} 
\usepackage{etoolbox}
%\usepackage{lmodern}
%\renewcommand\Authfont{\fontsize{12}{14.4}\selectfont}
%\renewcommand\Affilfont{\fontsize{9}{10.8}\itshape}
%\renewcommand\Authfont{\fontsize{12}{15}\selectfont}
%\renewcommand\Affilfont{\fontsize{9}{11}\itshape}
\definecolor{astral}{RGB}{46,116,181}
%\subsectionfont{\color{astral}}
%\sectionfont{\color{astral}}
%\usdate{17 May}                         % Use usual LaTeX date layout

%\title{\color{BlueViolet}\Huge{On the accuracy of approximated geodesic equations and different potentials with different numerical methods } }
\title{\color{BlueViolet}\Huge{Second order Field equations}}
%\vskip 2em
\author{Farbod Hassani}
%\thanks{Email:\href{mailto:farbod.hassani@unige.ch}{{farbod.hassani@unige.ch}}}  \thanks{Homepage: \href{http://www.farbod-hassani.com}{farbod-hassani.com}}}
%\affil{D\'epartement de Physique Th\'eorique and Center for Astroparticle Physics, Universit\'e de Gen\'eve,
%24 quai Ansermet, CH-1211 Gen\'eve 4, Switzerland}

%{farbod-hassani.com}} }
%\newcommand*{\TitleFont}{%     \usefont{\encodingdefault}{\rmdefault}{b}'%     \fontsize{18}{16}%    \selectfont}
%\title{\TitleFont Halo finder}
%\author[1]{{Farbod Hassani} \thanks{ \url{farbod.hassani@gmail.com}
%}
%\thanks{farbod-hassani.com}}
%\author[2]{Author E\thanks{E.E@university.edu}}
%\affil[1]{D\'epartement de Physique Th\'eorique and Center for Astroparticle Physics, Universit\'e de Gen\'eve,
%24 quai Ansermet, CH-1211 Gen\'eve 4, Switzerland}
%\emailAdd{farbod.hassani@gmail.com}
%\affil[2]{Department of Mechanical Engineering, \LaTeX\ University}
      %\begin{abstract}
%This is abstract text: This simple document shows very basic features of \LaTeX{}.
%\lstset { %
%    language=C++,
%    %backgroundcolor=\color{black!5}, % set backgroundcolor
%    basicstyle=\footnotesize,% basic font setting
%}
\begin{document}
  \maketitle
  \flushbottom
  
  \section{Metric and geometric quantities}
 {\large In order to implement in the Gevolution we need to change $\Phi$ and $\Psi$ Notation }. \\
  We take the metric in ADM form as below,
  \be
  ds^2= -N (t,\vec{x}) ^2 d t^2+ h_{ij } (t,\vec{x}) \Big( dx^i+N^i (t,\vec{x}) dt   \Big) \Big( dx^j+N^j (t,\vec{x}) dt   \Big)
  \ee
  where 
  \be
  N(t,\vec{x})= \bar{N} (t) e^{\epsilon \delta N (t,\vec{x})}
  \ee
  \be
  N^i=\epsilon \sigma^{0i} (t,\vec{x})
  \ee
  \be
    h_{ij}=a^2 \Big( e^{2 \zeta (t,\vec{x}) \epsilon} \delta_{ij} + \epsilon \sigma_{ij} (t,\vec{x})   \Big)
  \ee
  $\epsilon$ shows the order of terms in the scheme.\\
  For the first order equations we can define $\sigma_{ij}=(\partial_i \partial_j- \frac{\nabla^2}{3} \delta_{ij}) B(t,\vec{x}) \epsilon$ and $N^i=\delta ^{ij} \partial_j \psi (t,\vec{x}) \epsilon$  since we can separate the scalar, vector and tensor equations. \\
  On the other hand in second order equations we do observe the mixing of the scalar, vector and tensor equations according to $ T^{\mu \nu} \frac{\delta g_{\mu \nu}}{\delta (scalars)}$, which cannot be written as a derivative of a scalar equation and suggest other definition of the metric.  The calculation of $T^{\mu \nu} \frac{\delta g_{\mu \nu}}{\delta (scalars)}$ is presented in the appendix (?). \\
   \\
  { \color{red} {  For the second order equations we should think of writing the metric in general way, for example $\sigma^{ij}$ is a general function but How many? degrees of freedom in equations (scalar) or vector ...}}
\section{Definitions}
 The inverse of the metric is defined by the inverse of the  matrix.
\be
g^{. .}= ( g_{. .} )^{-1}
\ee
\be
N_i=h_{ij} N^j
\ee
Christoffel symbols:
\be
\Gamma_{\zeta \rho}^{\mu}= \frac{g^{\mu \xi }}{2} \left(  g_{\xi \zeta ,\rho }+ g_{\xi \rho ,\zeta } - g_{\rho \zeta ,\xi }   \right )
\ee
\be
K_{ij}=\frac{1}{2 N (t,\vec{x})} \left [  \dot{h}_{ij} - \nabla_{i} N_{j} - \nabla_{j} N_{i}  \right ]= \frac{1}{2 N (t,\vec{x})} \left [  \dot{h}_{ij} - \partial_{i} N_{j} - \partial_{j} N_{i} -2  \Gamma_{i j}^{l} N_l  \right ] 
\ee
\be
\delta K= K_i^i(t,\vec{x}) -\bar {K}_i^i (t)
\ee
Full metric,
\be
g_{00}= -N^2(t,\vec{x})+ h_{ij} N^i N^j \,,\, g_{ij}=h_{ij} \, , \, g_{0i}=g_{i0}=h_{ij}N^j
\ee
Riemann tensor
\be
R^{\rho}_{\sigma \mu \zeta}= \partial_{\mu} \Gamma_{\zeta \sigma}^{\rho}- \partial_{\zeta} \Gamma_{\mu \sigma}^{\rho} + \Gamma_{\mu \lambda}^{\rho} \Gamma_{\zeta \sigma}^{\lambda} -  \Gamma_{\zeta \lambda}^{\rho} \Gamma_{\mu \sigma}^{\lambda}
\ee
Ricci tensor;
\be
R_{\mu \rho}=R ^{\eta}_{\mu  \eta  \rho}
\ee
Ricci scalar;
\be
R=g^{\mu \rho} R_{\mu \rho}
\ee
 
 \subsection{Stuckelberg trick}
 { \color{red} {  Full calculation in the appendix?!}}
\be
f(t) \longrightarrow f(t) +  \dot{f} (t) \pi+ \frac{1}{2} \ddot{f }(t) \pi^2  + \frac{1}{6} \dddot{f }(t) \pi^3
\ee
\be
\Lambda(t) \longrightarrow \Lambda(t) +  \dot{\Lambda} (t) \pi+ \frac{1}{2} \ddot{\Lambda}(t) \pi^2 + \frac{1}{6} \dddot{\Lambda }(t) \pi^3
\ee
\be
M_2^4(t) \longrightarrow M_2^4(t) +  \dot{ M_2^4} (t) \pi+ \frac{1}{2}    \ddot{ M_2^4 }(t)  \pi^2 + \frac{1}{6} \dddot{M_2^4 }(t) \pi^3
\ee
\be
m_3^3(t) \longrightarrow m_3^3(t) +  \dot{m_3^3} (t) \pi+ \frac{1}{2} \ddot{m_3^3}(t) \pi^2 + \frac{1}{6} \dddot{m_3^3 }(t) \pi^3
\ee
\be
g^{00} \longrightarrow g^{00} + 2 g^{0 \mu} \partial_{\mu} \pi + g^{\rho \nu} \partial_{\rho} \partial_{\nu} \pi
\ee
\begin{align}
\delta K  \longrightarrow &   \delta K -3 \left ( \dot{H} \pi +\frac{1}{2} \ddot{H} \pi^2 \right ) - (1-\dot{\pi}) N h^{ij} \partial_i \partial_j \pi +\frac{1}{2} \partial_i h^{ij} \partial_j \pi  \nonumber \\ &+\frac{H}{2 a^2} \delta ^{ij}\partial_i \pi \partial_j \pi + \frac{2}{a^2} \delta ^{ij} \partial_i  \pi  \partial_j    \dot{\pi} -\frac{2}{a^2} \delta ^{ij} \partial_i N \partial_j \pi
\end{align}
{\color{red}The followed Stuckelberg trick is not true for second order, so one should write the Stuckelberg using the mathematica!}
The EFT action;
\be
S=\sqrt{-g} \left [ \frac{M_*^2}{2} f(t) R -\Lambda (t) -c(t) g^{00} +\frac{M_2^4(t)}{2} \left (g^{00} + \frac{1}{\bar{N}^2} \right )    -  \frac{m_3^3(t)}{2} \delta K  \left (g^{00} + \frac{1}{\bar{N}^2} \right )    \right ]
\ee



  Then the scalar field equations obtained by varying the action with respect to the scalars   { \color{red} { for second order equations  we need to decide for $N^i$ and $\sigma^{ij}$ instead of naively varying with respect to $\psi$ and B }}
    \be
 \frac{\delta S}{\delta \pi}=  ....
  \ee
    \subsection{Gauge transformation}
 Newtonian gauge:
 {\color{red}{Appendix!}}
 \be
 \delta N \rightarrow \Phi \, , \, \zeta \rightarrow-\Psi \, , \,  \psi \rightarrow0,  \,  \,  B \rightarrow0
 \ee
 
 \section{Field equations}
 By varying the action with respect to the scalars we get exactly the same first order equations of references ( {\color{red}{Ref!}}) and for the second order equations up to Gevolution's scheme we get,  
\subsection{$\pi$:}
%According to the obtained equation, there are no terms from
% $\{  -k_i k_j  \Phi ,-k_i k_j  \Psi,-k_i k_j  \pi , -k^2 \dot{\Phi},  -k^2 \dot{\Psi},-k^2 \dot{\pi} \\,  -i k^3  \Phi , -i k^3  \Psi,   -i k^3  \pi ,  k^4  \Phi ,  k^4  \Psi,  k^4  \pi  ,... \}$.  Again for $ \{  i \, \vec{k} \,  \Phi,i \vec{k}  \Psi,i \vec{k} \,    \pi  \}$ we have contribution only when they couple to each other, like $-k^2 \Phi \Psi$. So all possible terms  are chosen from  $ \{ \Phi, \Psi, \pi  \; , \dot{\Phi},\dot{\Psi},\dot{\pi} \; ,-k^2  \Phi,   -k^2  \Psi, -k^2  \pi  \}  $ plus three terms from $ \{  i \, \vec{k} \,  \Phi,i \vec{k}  \Psi,i \vec{k} \,    \pi  \}$  and double derivatives contributions . We write double time derivative terms $\ddot{\Phi} , \ddot{\Psi},\ddot{\pi} $ differently for simplicity with first order coefficients $B_{\ddot{\Phi}} (\Phi,\Psi,\pi), B_{\ddot{\Psi}} (\Phi,\Psi,\pi) , B_{\ddot{\pi}} (\Phi,\Psi,\pi) $ ;
\begin{align}
\frac{\delta S}{\delta \pi} &=  \int \int d^3k d^3 k' e^{i(\vec{k}+\vec{k}') . \vec{x}}  \Bigg [   -\frac{k^2}{a^2} C^{(2)}_{\Phi \Phi} \Phi \Phi  -\frac{k^2}{a^2} C^{(2)}_{\Phi \Psi} \Phi \Psi   -\frac{k^2}{a^2} C^{(2)}_{\Phi \pi} \Phi \pi 
  -\frac{k^2}{a^2} C^{(2)}_{\Psi \Phi} \Psi \Phi  -\frac{k^2}{a^2} C^{(2)}_{\Psi \Psi} \Psi \Psi   -\frac{k^2}{a^2} C^{(2)}_{\Psi \pi} \Psi \pi 
\nonumber  \\& 
   -\frac{k^2}{a^2} C^{(2)}_{\pi \Phi} \pi \Phi  -\frac{k^2}{a^2} C^{(2)}_{\pi \Psi} \pi \Psi   -\frac{k^2}{a^2} C^{(2)}_{\pi \pi} \pi \pi 
  -\frac{k^2}{a^2} C^{(2)}_{\dot{\Phi} \Phi} \dot{\Phi} \Phi  -\frac{k^2}{a^2} C^{(2)}_{\dot{\Phi} \Psi} \dot{\Phi} \Psi   -\frac{k^2}{a^2} C^{(2)}_{\dot{\Phi} \pi} \dot{\Phi} \pi 
\nonumber   \\&
 -\frac{k^2}{a^2} C^{(2)}_{\dot{\Psi} \Phi} \dot{\Psi} \Phi  -\frac{k^2}{a^2} C^{(2)}_{\dot{\Psi} \Psi} \dot{\Psi} \Psi   -\frac{k^2}{a^2} C^{(2)}_{\dot{\Psi} \pi} \dot{\Psi} \pi 
  +C_{ \dot{\pi}\dot{\pi}}  \dot{\pi}\dot{\pi}  -\frac{k^2}{a^2} C^{(2)}_{\dot{\pi} \Phi} \dot{\pi} \Phi  -\frac{k^2}{a^2} C^{(2)}_{\dot{\pi} \Psi} \dot{\pi} \Psi   -\frac{k^2}{a^2} C^{(2)}_{\dot{\pi} \pi} \dot{\pi} \pi \text{ .}
    \nonumber \\&
  -\frac{\vec{k}.\vec{k}'}{a^2}  C^{1,1}_{\Phi \Psi} \Phi \Psi  -\frac{\vec{k}.\vec{k}'}{a^2}  C^{1,1}_{\Phi \pi} \Phi \pi  -\frac{\vec{k}.\vec{k}'}{a^2}  C^{1,1}_{\Psi \pi} \Psi \pi  -\frac{\vec{k}.\vec{k}'}{a^2}  C^{1,1}_{\pi \pi} \pi \pi   -\frac{\vec{k}.\vec{k}'}{a^2}  C^{1,1}_{\Phi \Phi} \Phi \Phi -\frac{\vec{k}.\vec{k}'}{a^2}  C^{1,1}_{\Psi \Psi} \Psi \Psi    
  \nonumber \\ &
+ \vec{k} C^{(1)}_{\dot{\Phi}} (\Phi,\Psi,\pi) \dot{\Phi} + \vec{k}   C^{(1)}_{\dot{\Psi}} (\Phi,\Psi,\pi) \dot{\Psi}+ \vec{k}  C^{(1)}_{\dot{\pi}} (\Phi,\Psi,\pi) \dot{\pi} + C_{\ddot{\Phi}} (\Phi,\Psi,\pi) \ddot{\Phi} + C_{\ddot{\Psi}} (\Phi,\Psi,\pi) \ddot{\Psi}+ C_{\ddot{\pi}} (\Phi,\Psi,\pi) \ddot{\pi} \Bigg ] 
  \text{ .}
\end{align}
Some Important notes: \\
In order to write the equation we must write all the terms, since its possible we have made a mistake somewhere! (So all the equations which have been written up to now must change). \\
Be sure that all the terms are written, for example in the last notes I forgot to write the terms $\nabla^2 \Phi \dot{\pi}$, so write all the terms even if they are zero! Or just check in mathematica and write all non zero terms!
The red terms are where Filippo and I do not agree! so one of us has made a mistake! I wrote the Filippo's results in the red color terms! \\
This results are only variation with respect to field withour multiplying to $1/\sqrt{-g}$\\
By $C_{\ddot{\Psi}} $ I mean the coefficient of $\ddot{\Psi}$ and by  $C^{(1)}_{\dot{\Phi}}$ I mean the coefficient of $\partial_i \dot{\Phi}$ and $ C^{(2)}_{\dot{\Psi} \Phi}$ the coefficient of $\nabla^2 \Psi \Phi$. \\
Here we dont write the coefficient in Fourier space, so we dont need to multiply by $a^2$ or a minus sign!\\
So by the below I mean the coefficient of $\nabla^2 \Phi \Phi$
\be
 C^{(2)}_{\Phi \Phi}= -a M_*^2 \dot{f}\text{ .}
\ee
\be
C^{(2)}_{\Phi \Psi}= -a (m_3^3 - M_*^2 \dot{f}) \text{ .}
\ee
\be
C^{(2)}_{\Phi \pi}= a (\dot{m_3^3 } -M_*^2 \ddot{f}  )     \text{ .}
\ee
\be
C^{(2)}_{\Phi \dot{\Phi}}=   0     \text{ .}
\ee
\be
C^{(2)}_{\Phi  \dot{\Psi}}= 0   \text{ .}
\ee
\be
 C^{(2)}_{\Phi  \dot{\pi}}= a m_3^3    \text{ .} 
\ee
\be
C^{(2)}_{\Psi \Phi}= a (2M_*^2 \dot{f})  \text{ .}
\ee
Something is wrong in my new calculation (Stuckelberg trick of K, I get (-1+$a^2$) but it should be 0.
\be
C^{(2)}_{\Psi \Psi}=a (2M_*^2 \dot{f})  \text{ .}
\ee
\be
C^{(2)}_{\Psi \pi}= a (2M_*^2 \ddot{f})   \text{ .}
\ee
\be
C^{(2)}_{\Psi \dot{\Phi}}=   0     \text{ .}
\ee
\be
C^{(2)}_{\Psi  \dot{\Psi}}= 0   \text{ .}
\ee
\be
C^{(2)}_{\Psi  \dot{\pi}}=0    \text{ .} 
\ee
Again in top the sam easy problem ($-1+a^2$)
\be
C^{(2)}_{\pi \Phi}= a (-3 m_3^3 H +2c -4 M_2^4 + \dot{m_3^3})  \text{ .}
\ee
\be
C^{(2)}_{\pi \Psi}= a ( -m_3^3 H -2c - \dot{m_3^3}) \text{ .}
\ee
\be
C^{(2)}_{\pi \pi}=  a (\dot{m_3^3} H + 2\dot{c} -3 m_3^3 \dot{H} + \ddot{m_3^3} ) \text{ .}
\ee
\be
C^{(2)}_{\pi \dot{\Phi}}=   -a m_3^3     \text{ .}
\ee
\be
C^{(2)}_{\pi  \dot{\Psi}}= -4 a m_3^3   \text{ .}
\ee
\be
 C^{(2)}_{\pi  \dot{\pi}}=4 a M_2^4    \text{ .} 
\ee
\be
C^{(2)}_{\dot{\Phi} \Phi}= 0\text{ .}
\ee
\be
C^{(2)}_{\dot{\Phi} \Psi}=0 \text{ .}
\ee
\be
C^{(2)}_{\dot{\Phi} \pi}= 0 \text{ .}
\ee
\be
C^{(2)}_{\dot{\Phi} \dot{\Phi}}= 0     \text{ .}
\ee
\be
C^{(2)}_{\dot{\Phi}  \dot{\Psi}}= 0\text{ .}
\ee
\be
 C^{(2)}_{\dot{\Phi}  \dot{\pi}}=  0  \text{ .} 
\ee
\be
C^{(2)}_{\dot{\Psi} \Phi}=0 \text{ .}
\ee
\be
C^{(2)}_{\dot{\Psi} \Psi}=0 \text{ .}
\ee
\be
C^{(2)}_{\dot{\Psi} \pi}= 0 \text{ .}
\ee
\be
C^{(2)}_{\dot{\Psi} \dot{\Phi}}= 0     \text{ .}
\ee
\be
C^{(2)}_{\dot{\Psi}  \dot{\Psi}}= 0\text{ .}
\ee
\be
 C^{(2)}_{\dot{\Psi}  \dot{\pi}}=  0  \text{ .} 
\ee
\be
C^{(2)}_{\dot{\pi} \Phi}=0   \text{ .}
\ee
\be
C^{(2)}_{\dot{\pi} \Psi}= 0 \text{ .}
\ee
\be
C^{(2)}_{\dot{\pi} \pi}=0 \text{ .}
\ee
\be
C^{(2)}_{\dot{\pi} \dot{\Phi}}= 0     \text{ .}
\ee
\be
C^{(2)}_{\dot{\pi}  \dot{\Psi}}= 0\text{ .}
\ee
\be
 C^{(2)}_{\dot{\pi}  \dot{\pi}}=  0  \text{ .} 
\ee
\be
C^{1,1}_{\Phi \Phi}=-a M_*^2 \dot{f} \text{ .} 
\ee
\be
C^{1,1}_{\Phi \Psi}= a (-m_3^3 +M_*^2\dot{f})\text{ .}
\ee
\be
 C^{1,1}_{\Phi \pi}=2 a (c-2M_2^4 -H m_3^3 +\dot{m_3^3})   .   \text{ .}
\ee
\be
 C^{1,1}_{\Psi \pi}=a ( -2c-H m_3^3 -\dot{m_3^3})     \text{ .} 
\ee
I see a difference with Filippo in the last equation,
\be
C^{1,1}_{\Psi \Psi}=3a M_*^2 \dot{f} \text{ .}
\ee
\be
C^{1,1}_{\pi \pi}=-\frac{a}{2}  \Big (m_3^3 H^2 +4 m_3^3 \dot{H} -2 \dot{c} -4 \dot{M_2^4} -4 M_2^4 H + \dot{m_3^3} H    \Big ) \text{ .} 
\ee

\be
C^{(1)}_{\dot{\Phi}} (\Phi,\Psi,\pi)= a \Big(a^2 \big[2c +4 M_2^4+ 3 H (m_3^3-M_*^2 \dot{f}) \big] -m_3^3 \nabla^2 \pi \Big)
\ee
\be
C^{(1)}_{\dot{\Psi}} (\Phi,\Psi,\pi)=a \Big ( 3a^2 \big ( 2c +3 H m_3^3 -4 H M_*^2 \dot{f} + \dot{m_3^3} \big ) -4 m_3^3 \nabla^2 \pi   \Big)
\ee
\be
C^{(1)}_{\dot{\pi}} (\Phi,\Psi,\pi)= a \Big( -2 a^2 \big(3 c H + 6 H m_2^4 +\dot{c}+ 2 \dot{m_2^4}  \big) + 4m_2^4 \nabla^2 \pi + m_3^3 \nabla^2 \Phi   \Big)   
\ee
\be
C_{\ddot{\Phi}} (\Phi,\Psi,\pi)=0	\text{ .}
\ee
\be
C_{\ddot{\Psi}} (\Phi,\Psi,\pi)= 3 a^3 (m_3^3 - M_*^2 \dot{f}) \text{ .} 
\ee
Again we get different.
\be
C_{\ddot{\pi}} (\Phi,\Psi,\pi)=  a (-2 a^2 (c+2 m_2^4) + m_3^3 \nabla^2 \pi)	\text{ .} 
\ee
Check again the top equation.

\end{document}
 